

\section{MIPS instruction-set specification}
This section is a complete specification of the MIPS instruction set.
The MIPS is a good machine to start with
because it has a small, regular instruction set.
Its specification
requires only the toolkit's basic features, and the regularity of the
instruction set lends itself to factoring, making the specification
small. 
As in the MIPS manual~\cite{kane:mips}, the specification is divided into
two parts: one describing the instructions of the main CPU, the other
describing those of the floating-point coprocessor.

\subsubsection*{Tokens and fields}

The MIPS is a RISC machine; all instructions occupy exactly one 
32-bit token.  
Although the instructions of the floating-point coprocessor are listed
separately in the manual, they use some of the same fields as the
instructions of the main CPU.
This property makes it possible to define all the fields in a single 
token class.
We use {\tt noweb} to distribute the field specifications, putting 
most in the CPU section and the rest in the coprocessor section.

 [[fields]] specifications give only the positions of fields.
 [[fieldinfo]] specifications give more information;
for example, we use special names to refer to the values of
fields that denote registers.
[[fieldinfo]] specifications are never required.

The core MIPS specification 
describes the fields first, then patterns and
constructors.
We tell {\tt noweb} to put it in a file called {\tt mips-core.spec}.
<<mips-core.spec>>=
fields of instruction (32) <<field specifications>>
<<[[fieldinfo]] specifications>>
<<pattern and constructor specifications>>
@
\subsection{The MIPS CPU}

A field is specified by giving a range of bit positions. 
Some of the MIPS fields are listed on page~A-3 of the MIPS
architecture manual; others, like [[breakcode]], apply to only 
a couple of instructions.
The MIPS manual sometimes uses more than one name for the same field, for
example, [[offset]] and [[base]] are used instead  of [[immed]] and [[rs]] to
describe load and store instructions.
The numbers in the specification
are the starting and ending bit positions,
where 0 is the least and 31 the most significant bit.
<<field specifications>>=
word 0:31 op 26:31 rs 21:25 rt 16:20 immed 0:15 offset 0:15 base 21:25 
target 0:25 rd 11:15 shamt 6:10 funct 0:5 cond 16:20 breakcode 6:25
@
Special names are associated with the integer registers.
These names can be used in constructor applications, as shown 
in the definitions of synthetic instructions in
Section~\ref{sec:synth}.
These names might also one day be used to generate symbolic assemblers and
disassemblers from a toolkit specification.
On the MIPS, we use names of the form $\hbox{\tt r}n$, except for the
stack pointer ([[sp]]).
<<properties of integer-register fields>>=
names [ r0  r1  r2  r3  r4  r5  r6  r7  r8  r9  r10 r11 r12 r13 r14 r15
        r16 r17 r18 r19 r20 r21 r22 r23 r24 r25 r26 r27 r28 sp  r30 r31 ]
@ These names are just one of the properties that could be associated
with the integer registers, which are [[rs]], [[rd]], [[rt]], and [[base]].
<<[[fieldinfo]] specifications>>=
fieldinfo [ rs rt rd base ] is [ <<properties of integer-register fields>> ]
@
\subsubsection{Opcodes}
An opcode is a pattern that constrains the values of 
one or more fields.
Defining opcodes individually would be tedious, and the result would
be hard to compare with the architecture manual, which uses
opcode tables.
The [[patterns]] declaration binds a table of names when
a {\em generating expression} appears on the right-hand side, letting
us write declarations that resemble the tables in the MIPS manual.
 
The numeric codes for the MIPS opcodes are described in three
tables on page~A-87 of the MIPS architecture manual.
These tables are reproduced in Figure~\ref{figure:mips-tables}.
Normal opcodes are six bits, and they appear in the [[op]] field of the
instruction.
This statement names patterns for each of the opcodes in the table at
the top of Figure~\ref{figure:mips-tables}.
<<pattern and constructor specifications>>=
patterns
 [ special bcond   j       jal     beq     bne     blez    bgtz
   addi    addiu   slti    sltiu   andi    ori     xori    lui
   cop0    cop1    cop2    cop3    _       _       _       _
   _       _       _       _       _       _       _       _
   lb      lh      lwl     lw      lbu     lhu     lwr     _
   sb      sh      swl     sw      _       _       swr     _
   lwc0    lwc1    lwc2    lwc3    _       _       _       _
   swc0    swc1    swc2    swc3    _       _       _       _ ] 
 is op = {0 to 63}
@ This statement creates names for patterns constraining the [[op]]
field.
The expression ``[[op = {0 to 63}]]'' generates a list of 64~patterns,
ranging from ``[[op = 0]]'' to ``[[op = 63]]''.
Each pattern is associated with the corresponding name in the list to
the left-hand side of [[is]].
For example, an instruction matches the pattern [[beq]] if its [[op]] field
is~4.
Patterns associated with the name ``[[_]]'' are discarded.

\begin{figure}
{\tabskip=0pt\offinterlineskip\footnotesize
\newbox\strutbox
\def\strut{\vrule height 8pt depth 3pt width0pt}
\def\line{&\multispan{17}\hrulefill}
\noindent\halign{&\hbox to 5em{\hss{#}\hss}&\strut\vrule #\cr
&\omit&28..26&\multispan{12}\hfil\normalsize\bf Opcode\hfil&\omit&\omit&\omit\cr
\strut 31..29&
\omit&\scriptsize0&\omit&\scriptsize1&\omit&\scriptsize2&\omit&\scriptsize3&
\omit&\scriptsize4&\omit&\scriptsize5&\omit&\scriptsize6&\omit&\scriptsize7&\omit\cr
\line\cr \scriptsize0&&SPECIAL&&BCOND&&J&&JAL&&BEQ&&BNE&&BLEZ&&BGTZ&\cr
\line\cr \scriptsize1&&ADDI&&ADDIU&&SLTI&&SLTIU&&ANDI&&ORI&&XORI&&LUI&\cr
\line\cr \scriptsize2&&COP0&&COP1&&COP2&&COP3&&\dag&&\dag&&\dag&&\dag&\cr
\line\cr \scriptsize3&&\dag&&\dag&&\dag&&\dag&&\dag&&\dag&&\dag&&\dag&\cr
\line\cr \scriptsize4&&LB&&LH&&LWL&&LW&&LBU&&LHU&&LWR&&\dag&\cr
\line\cr \scriptsize5&&SB&&SH&&SWL&&SW&&\dag&&\dag&&SWR&&\dag&\cr
\line\cr \scriptsize6&&LWC0&&LWC1&&LWC2&&LWC3&&\dag&&\dag&&\dag&&\dag&\cr
\line\cr \scriptsize7&&SWC0&&SWC1&&SWC2&&SWC3&&\dag&&\dag&&\dag&&\dag&\cr
\line\cr
\noalign{\bigskip}
&\omit&2..0&\multispan{12}\hfil\normalsize\bf SPECIAL\hfil&\omit&\omit&\omit\cr
\strut 5..3&
\omit&\scriptsize0&\omit&\scriptsize1&\omit&\scriptsize2&\omit&\scriptsize3&
\omit&\scriptsize4&\omit&\scriptsize5&\omit&\scriptsize6&\omit&\scriptsize7&\omit\cr
\line\cr \scriptsize0&&SLL&&\dag&&SRL&&SRA&&SLLV&&\dag&&SRLV&&SRAV&\cr
\line\cr \scriptsize1&&JR&&JALR&&\dag&&\dag&&SYSCALL&&BREAK&&\dag&&\dag&\cr
\line\cr \scriptsize2&&MFHI&&MTHI&&MFLO&&MTLO&&\dag&&\dag&&\dag&&\dag&\cr
\line\cr \scriptsize3&&MULT&&MULTU&&DIV&&DIVU&&\dag&&\dag&&\dag&&\dag&\cr
\line\cr \scriptsize4&&ADD&&ADDU&&SUB&&SUBU&&AND&&OR&&XOR&&NOR&\cr
\line\cr \scriptsize5&&\dag&&\dag&&SLT&&SLTU&&\dag&&\dag&&\dag&&\dag&\cr
\line\cr \scriptsize6&&\dag&&\dag&&\dag&&\dag&&\dag&&\dag&&\dag&&\dag&\cr
\line\cr \scriptsize7&&\dag&&\dag&&\dag&&\dag&&\dag&&\dag&&\dag&&\dag&\cr
\line\cr
\noalign{\bigskip}
&\omit&18..16&\multispan{12}\hfil\normalsize\bf BCOND\hfil&\omit&\omit&\omit\cr
\strut 20..19&
\omit&\scriptsize0&\omit&\scriptsize1&\omit&\scriptsize2&\omit&\scriptsize3&
\omit&\scriptsize4&\omit&\scriptsize5&\omit&\scriptsize6&\omit&\scriptsize7&\omit\cr
\line\cr \scriptsize0&&BLTZ&&BGEZ&&\multispan{11}\hfil&&\cr
\line\cr \scriptsize1&&&&&&\multispan{11}\hfil&&\cr
\line\cr \scriptsize2&&BLTZAL&&BGEZAL&&\multispan{11}\hfil&&\cr
\line\cr \scriptsize3&&\multispan{3}&&\multispan{11}\hfil&&\cr
\line\cr
\noalign{\bigskip{\leftskip=3em\advance\hsize by 0.8in \noindent\llap{\dag\ }%
  Operation codes marked with a dagger cause reserved instruction exceptions 
  and are reserved\strut\hfil\break\strut for future versions of the architecture.\par}}
}}
\caption{Opcode tables from MIPS architecture manual.}
\label{figure:mips-tables}
\end{figure}


@
%\fillup{1.8in}
Two opcodes, [[special]] and [[bcond]], are used for several instructions.
These instructions are decoded by checking the bit-pattern in the
[[funct]] and [[cond]] fields of the instructions, respectively.
The following statement names patterns for each of the [[special]] opcodes in the table 
in the center of Figure~\ref{figure:mips-tables}:
<<pattern and constructor specifications>>=
patterns
 [ sll     _       srl     sra     sllv    _       srlv    srav
   jr      jalr    _       _       syscall break   _       _
   mfhi    mthi    mflo    mtlo    _       _       _       _
   mult    multu   div     divu    _       _       _       _
   add     addu    sub     subu    and     or      xor     nor
   _       _       slt     sltu    _       _       _       _ ] 
 is  special & funct = {0 to 47}
@
The [[bcond]] table at the bottom of Figure~\ref{figure:mips-tables}
is sparse.
Instead of writing a table with many empty entries,
we use another form of generating expression on the right-hand side;
the list of integers in square brackets
generates each integer in turn.
<<pattern and constructor specifications>>=
patterns
 [ bltz bgez bltzal bgezal ] is bcond & cond = [ 0 1 16 17 ]
@ These statements create new patterns by adding further constraints to
the existing patterns [[special]] and [[bcond]].

A slightly different way to look at this style of specification is
that ``[[special]]'' and ``[[bcond]]'' are the names of tables, not
opcodes.  As we'll see, that view is a better one for the SPARC.



@
\subsubsection{Grouping instructions}
The rest of the patterns organize instructions into groups, much
as is done in Chapter~3 of the MIPS manual.
Most instructions are grouped by assembly-language syntax.
Immediate-mode instructions are grouped
into signed and unsigned variants ([[immedS]] and [[immedU]]), 
depending on whether the immediate operand
is  sign-extended.
The toolkit uses disjunction to define patterns
that match any of a group of related instructions.
<<pattern and constructor specifications>>=
patterns
  load    is lb | lbu | lh | lhu | lw | lwl | lwr | sb | sh | sw | swl | swr 
  immedS  is addi | addiu | slti | sltiu
  immedU  is andi | ori | xori 
  arith3  is add | addu | sub | subu | slt | sltu | and | or | xor | nor 
  shift   is sll | srl | sra
  vshift  is sllv | srlv | srav 
  arith2  is mult | multu | div | divu
  mfhilo is mfhi | mflo 
  mthilo is mthi | mtlo
  jump    is j  | jal
  jumpr   is jr | jalr
  branch1 is blez | bgtz | bltz | bgez | bltzal | bgezal
  branch2 is beq | bne
  copls   is lwc0 | lwc1 | lwc2 | lwc3 | swc0 | swc1 | swc2 | swc3
@
\subsubsection{Constructors for simple instructions}
The patterns that group instructions
can now be used to define constructors.
A constructor specification contains an opcode, operands, possibly a set of equations, 
and an output pattern.
The operands correspond to the procedure's arguments, and when the procedure is called,
it emits the sequence of tokens defined by the output pattern.
The toolkit generates an encoding procedure for each constructor.
When the constructor name on the left-hand side is the name of a pattern,
each disjunct of the pattern is used to generate a constructor.
For example, the first declaration below generates a constructor 
for each disjunct bound to the pattern [[load]], i.e.,
[[lb]], [[lbu]], [[lh]], etc.

Most output patterns for the MIPS constructors are conjunctions of
the opcode and all the operands. 
For example, the load instructions could be described by
\begin{quote}
[[load rt, offset!(base) is load & rt & offset & base]].
\end{quote} 
Because this idiom is so common, not only on the MIPS but on other
machines as well, the toolkit provides a special abbreviation for it:
if the output pattern is omitted, the conjunction is inferred.
We can use this trick to specify almost all of the integer
instructions; only the one- and two-operand branches ([[branch1]] and
[[branch2]]) and the jumps need to be treated specially.
Given the groupings we defined above, we specify 56~instructions in
just 15~lines:%
\footnote{This is the one place in specifications where newlines can be
significant, preventing the opcode of a following pattern be taken as an
operand of the current pattern.  The reference manual explains the
lexical details.}
<<pattern and constructor specifications>>=
constructors
  load   rt, offset!(base)
  immedS rt, rs, offset!
  immedU rt, rs, offset
  lui    rt, offset
  arith3 rd, rs, rt
  shift  rd, rt, shamt
  vshift rd, rt, rs
  arith2 rs, rt
  mfhilo rd
  mthilo rs
  syscall
  break  breakcode
  copls  ft, offset!(base)
  jr     rs
  jalr   rd, rs
@ Operands with a trailing [[!]] (e.g., [[offset!]]) are treated as
signed parameters in encoding procedures; the toolkit implicitly
narrows their values to produce unsigned fields.
@
These constructor specifications include
the assembly-language syntax associated with the instructions being
specified.
For example, most operand are separated by commas, but the syntax
[[offset(base)]] is used to suggest the address computation in the
load and store instructions. 
Using assembly syntax in the constructor
specification makes the specification look more like the architecture
manual. 
We hope one day to use the assembly syntax to produce
an assembler and disassembler from this specification.
@

\subsubsection{Branches and relocatable addresses}

Because it's a RISC machine, the MIPS can't use target addresses
directly as field values in branch instructions; a 32-bit address
wouldn't leave room for an opcode, or anything else.
Instead, the branches are relative to the program counter; the field
corresponding to the target address records the difference between that
address and the PC.
We use equations to specify the relationships between the operands and
fields.
<<pattern and constructor specifications>>=
<<placeholder definition>>
relocatable reloc
constructors
  branch1 rs, reloc
     { reloc = L + 4 * offset! } is branch1 & rs & offset; L: epsilon
  branch2 rs, rt, reloc
     { reloc = L + 4 * offset! } is branch2 & rs & rt & offset; L: epsilon
@
The branch equations capture the semantics of the MIPS branch instructions;
the offset is sign-extended, shifted left 2 bits, and added to the
address of the delay slot, i.e., [[L]]. 
@
Constructor operands whose name begin with [[reloc]]
are {\em relocatable addresses}, like the branch targets above.
Pattern labels, like [[L]] above, are also relocatable addresses.
When a constructor that uses relocatable addresses is applied, it
checks to see if those addresses are known (i.e., they have
been assigned absolute adresses). 
If so, it treats them as ordinary integers and emits the 
instruction. 
Otherwise, it emits {\em placeholder} tokens and creates
a relocation closure. 
The placeholders will be overwritten later, when the address becomes
known and the application applies the closure.
@
We have to specify a placeholder pattern for tokens of each class.
On the MIPS, we have only one class, so we need one placeholder.
We've picked a [[break]] instruction with code~99, so that we'll get a trap
if by some mischance the placeholder sneaks into a real program and we
try to execute it.
<<placeholder definition>>=
placeholder for instruction is break(99)
@ The syntax ``[[break(99)]]'' isn't a special thing we use for
placeholders;  it's just the pattern we get by applying the [[break]] constructor
to the value 99.
@

The PC-relative branches can't jump anywhere in the MIPS address
space, only within $2^{16}$ words of the PC.
The MIPS provides a [[jump]] instruction that lets a program jump to
any word in the same quarter of the address space as the PC.
It's best specified using a lot of bit operations; the toolkit
provides a notation for taking {\em bit slices} of values.
For example, [[reloc[28:31]]] denotes the four most significant bits
of the 32-bit, relocatable address [[reloc]].
The description of [[j]] on page~A-31 in the MIPS manual corresponds
to the equations below.
<<pattern and constructor specifications>>=
constructors
  jump reloc { reloc@[28:31] = L@[28:31], 
               reloc@[2:27]  = target, 
               reloc@[0:1]   = 0 } 
         is L: jump & target
@ The high bits of the target address must match the current address, the middle
bits give the [[target]] field, and the low bits must be zero.
@
\subsubsection{Synthetic instructions and conditional assembly}
\label{sec:synth}
We have completed our description of the MIPS integer instructions,
but MIPS 
applications often use ``synthetic'' instructions, 
i.e., instructions that are in the assembly
language but not in the hardware instruction set.
We specify constructors for synthetic instructions by defining them in
terms of existing constructors.
We put synthetic instructions in a separate specification, 
because not all applications need synthetics.
<<mips-synth.spec>>=
constructors
  nop         	is  sll (r0, r0, 0)
  mov rd, rs    is  addu(rd, rs, r0)
  b reloc     	is  beq (r0, r0, reloc)
@ The earlier [[fieldinfo]] specification permits us to use
names, not numbers, to refer to register values.
@
There is a large collection of synthetic branch instructions; they
combine comparisons and branches.  The result of comparisons is left
in register~1, which MIPS conventions reserve for the assembler.
<<mips-synth.spec>>=
constructors
  bge  rs, rt, reloc  is  slt (r1, rs, rt);  beq(r1, r0, reloc)
  bgeu rs, rt, reloc  is  sltu(r1, rs, rt);  beq(r1, r0, reloc)
  blt  rs, rt, reloc  is  slt (r1, rs, rt);  bne(r1, r0, reloc)
  bltu rs, rt, reloc  is  sltu(r1, rs, rt);  bne(r1, r0, reloc)
  bleu rs, rt, reloc  is  sltu(r1, rt, rs);  beq(r1, r0, reloc)
  ble  rs, rt, reloc  is  slt (r1, rt, rs);  beq(r1, r0, reloc)
  bgt  rs, rt, reloc  is  slt (r1, rt, rs);  bne(r1, r0, reloc)
  bgtu rs, rt, reloc  is  sltu(r1, rt, rs);  bne(r1, r0, reloc)
@
A full multiplication requires fetching the result from a special register.
<<mips-synth.spec>>=
constructors
  mul rd, rs, rt  is  multu(rs, rt); mflo(rd)
@
Sometimes the best expansion for a synthetic 
instruction depends on the values of operands.
We can choose one of several expansions by putting alternatives
on the right-hand side of a constructor specification,
each with its own set of equations.
Each application of the constructor uses the first alternative for
which the equations can be solved.
For example, the [[li]] (load immeditate) synthetic instruction has
three ways to load a signed value [[imm]] into register [[rt]].
When [[imm]] fits in 16 bits, use an immediate mode [[addiu]] instruction
where the second operand is register 0, which is always 0.
When the low-order 16 bits of [[imm]] are zero, 
use [[lui]] to assign the high-order bits to [[rt]];
[[lui]] automatically zero's the low-order bits.
In the general case, cut [[imm]] into slices and use two instructions:
[[lui]] to assign the high-order bits and [[addiu]] to add in the low-order
bits.
<<mips-synth.spec>>=
constructors
  li rt, imm  when { imm@[16:31]! = imm@[15]! } is addiu(rt, r0, imm@[0:15]!)
              when { imm@[0:15] = 0 }           is lui  (rt, imm@[16:31])
              otherwise is lui(rt, imm@[16:31] + imm@[15]); addiu(rt, rt, imm@[0:15]!)
@ For the third branch, the toolkit generates an unnecessary test to
see that [[imm@[16:31] + imm@[15]]] doesn't overflow 16~bits.
The test isn't needed, because the only way it can happen is when
everything in sight is ones, and in that case we take the first
branch.
I'd love to have a symbolic-algebra system smart enough to figure this out.
@
\subsection{The floating-point coprocessor}

Pages B-5 through B-7 of the MIPS manual introduce a few more field names for the
convenience of specifying the floating-point instructions.
The integer-register fields [[rd]], [[rs]], and [[rt]] could be re-used to refer
to fields of floating-point instructions, but we introduce new fields
[[fd]], [[fs]], and [[ft]] in order to have different names for the registers.
<<field specifications>>=
ft 16:20 fs 11:15 fd 6:10 format 21:24 bit25 25:25
<<[[fieldinfo]] specifications>>=
fieldinfo [ fs ft fd ] is [ <<properties of floating-point-register fields>>]
<<properties of floating-point-register fields>>=
names [ f0  f1  f2  f3  f4  f5  f6  f7  f8  f9  f10 f11 f12 f13 f14 f15
        f16 f17 f18 f19 f20 f21 f22 f23 f24 f25 f26 f27 f28 f29 f30 f31 ]
@
The [[format]] field has only three legitimate values: 0, 1, and 4.
The [[sparse]] field attribute associates the names 
[[s]], [[d]], and [[w]] (signifying single, double, and word,
respectively) with these values.
<<[[fieldinfo]] specifications>>=
fieldinfo format is [ sparse [ s = 0, d = 1, w = 4 ] ]
@ 
The instruction codes for Coprocessor 1 (floating point)
are given on page B-28 of the MIPS manual.
We use names like ``[[add.]]'' instead of ``[[add.fmt]]'' because 
they make it possible to form the full name of the instruction by
concatenating the name of the opcode pattern and the name of the format,
e.g., [[add.s]], [[add.d]], etc.
They also distinguish the floating-point opcodes from analogous
integer opcodes.
<<pattern and constructor specifications>>=
patterns
 [ add.    sub.    mul.    div.    _       abs.    mov.    neg.   
   _       _       _       _       _       _       _       _
   _       _       _       _       _       _       _       _
   _       _       _       _       _       _       _       _
   cvt.s   cvt.d   _       _       cvt.w   _       _       _
   _       _       _       _       _       _       _       _
   c.f     c.un    c.eq    c.ueq   c.olt   c.ult   c.ole   c.ule
   c.sf    c.ngle  c.seq   c.ngl   c.lt    c.nge   c.le    c.ngt ] 
 is cop1 & funct = {0 to 63} & bit25 = 1
@

Specifying the branch and move instructions is more complicated, 
because the relevant codes span several entries in the table,
and the pattern language is designed to bind one pattern to one entry.
We introduce two new fields to simplify the specification.
<<field specifications>>=
cop1code 22:25 copbcode 16:16
<<pattern and constructor specifications>>=
patterns
  [ mfc1 cfc1 mtc1 ctc1 ] is cop1 & cop1code = {0 to 3} & funct = 0
  bc1x is cop1 & (cop1code = 4 | cop1code = 6)
  bc1f is bc1x & copbcode = 0
  bc1t is bc1x & copbcode = 1
@
The pattern [[bc1x]] is under-constrained.
The architecture manual indicates that the [[cop1code]] field
for the [[bc1f]] and [[bc1t]] instructions can have the value
4 or 6.
When generating encoding code for the [[bc1f]] and [[bc1t]]
constructors, the toolkit warns that their output patterns
are under-constrained and chooses the [[cop1code = 4]], as described in the
first disjunct for [[bc1x]].
Toolkit-generated decoding code recognizes either variant.

@
{\hfuzz=0.7pt
Like the integer opcodes, 
the floating-point opcodes are grouped by assembly-language syntax.
\par}
<<pattern and constructor specifications>>=
patterns
  arith3. is add. | div. | mul. | sub.
  arith2. is abs. | mov. | neg. 
  movec1  is mfc1 | mtc1 | cfc1 | ctc1
  c.cond  is c.f  | c.un   | c.eq  | c.ueq | c.olt | c.ult | c.ole | c.ule |
             c.sf | c.ngle | c.seq | c.ngl | c.lt  | c.nge | c.le  | c.ngt  
  lsc1    is lwc1 | swc1
  convert is cvt.s | cvt.d | cvt.w

@ 
The following constructor specifications introduce the compound-opcode 
concatenation operator ([[^]]).
Each floating-point opcode has three variants, according to whether the value of the 
[[format]] field is [[s]], [[d]], or [[w]].
We form the constructors by concatenating opcode names with [[format]].
One constructor is created for each member of the cross product of the patterns named,
(e.g., [[add.s]], [[add.d]], [[add.w]], [[div.s]], [[div.d]], etc).
Constructor names can also include string literals delimited
by double quotes, e.g. [[convert^"."^format]].
As with the integer instructions, most output patterns are implicit.

The floating-point arithmetic, conditional, and type-conversion
instructions all require floating-point operands that are even-odd
register pairs.  The equations for [[arith3.^format]], etc., below
specify that the register operands are all even values. 
These equations have been commented out because we believe people
won't want to pay the overheads of checking when generating encoding procedures,
but we want them when we validate the spec, so I have arranged
for the command \mbox{[[sed 's/# {/{/']]} to uncomment them.
<<pattern and constructor specifications>>=
constructors
  arith3.^format fd, fs, ft # { fd = 2 * _, fs = 2 * _, ft = 2 * _ }
  arith2.^format fd, fs     # { fd = 2 * _, fs = 2 * _ }
  movec1         rt, fs     
  c.cond^"."^format  fs, ft # { fs = 2 * _, ft = 2 * _ }
  convert^"."^format fd, fs # { fd = 2 * _, fs = 2 * _ }
  "bc1f" reloc { reloc = L + 4 * offset! } is bc1f & offset; L: epsilon
  "bc1t" reloc { reloc = L + 4 * offset! } is bc1t & offset; L: epsilon
@ We have to quote [[bc1f]] and [[bc1t]] instead of using them
directly as opcodes, because these patterns have multiple
disjuncts.  The constructor-specification process ``explodes''
opcodes, so it would pick just one disjunct, and we would be unable to
recognize the other when decoding an application of the constructor.
@
There are several synthetic instructions that operate
on pairs of floating point registers.
[[l.d]] and [[s.d]] load and store double words, respectively. 
[[mtc.d]] and [[mfc.d]] move pairs of words in registers
to and from the floating point co-processor.
<<mips-synth.spec>>=
constructors
  l.d ft,offset!(base) # { ft = 2 * _ } 
      is  lwc1(ft, offset!, base); lwc1(ft+1, offset!+4, base)
  s.d ft,offset!(base) # { ft = 2 * _ } 
      is  swc1(ft, offset!, base); swc1(ft+1, offset!+4, base)
  mtc1.d rt, fs  # { rt = 2 * _, fs = 2 * _ } 
      is  mtc1(rt, fs); mtc1(rt+1, fs+1)
  mfc1.d rt, fs  # { rt = 2 * _, fs = 2 * _ } 
      is  mfc1(rt, fs); mfc1(rt+1, fs+1)
@ Because [[offset]] is a field, it is unsigned, and so we must use
[[offset!]] to get the signed offset that [[lwc1]] and similar
constructors expect.


\iffalse
\subsubsection{Other synthetic instructions}

We include one more synthetic instruction to illustrate
the use of the sequence ([[;]]) pattern operator.
[[trunc.w.d]] implements the [[trunc.w.d]] assembly instruction
and requires a sequence of 13 tokens.
When applied, the [[trunc.w.d]] constructor emits the complete
sequence specified on the right-hand side.

We've put this thing in its own specification file because it doesn't
appear in the TR (plus this way we'll be able to elide using an elide
filter if ever I write one).
<<mips-trunc.spec>>=
constructors
  trunc.w.d fs, ft, rt is 
    cfc1(rt, f31); 
    cfc1(rt, f31);   nop(); 
    ori(r1, rt, 3);
    xori(r1, r1, 2); 
    ctc1(r1, f31); 
    srlv(r0, r0, r0);
    cvt.w.d(fs, ft); nop();
    ctc1(rt, f31);   nop(); 
    mfc1(rt, fs);    nop()
@ \fi 

@
\subsection{Application-specific specifications for the {\tt mld} linker}

An application writer has the option of extending a machine's specification
to include application-specific information.
These extensions can simplify implementation of the application or
can be used by the toolkit to generate more efficient
encoding code.
{\tt mld} is a retargetable, optimizing linker that generates code for
the MIPS or the SPARC.
{\tt mld} uses the toolkit to emit executable binary directly, instead
of going through the assembler.
This means it needs substitutes for the synthetic [[div]], [[divu]], [[rem]],
and [[remu]] instructions provided by the MIPS assembler.
@
The MIPS hardware does no checking for divide by zero or divisions
that overflow; these checks have to be done in software.
A [[break 7]] instruction is used to indicate one of these errors; so
{\tt mld} has to generate code that does these tests.
The first test just checks to see if the divisor is zero, branching
around the break instruction if it isn't.
If, for some reason, we can tell at compile time that the
divisor is zero, we just emit a break, because that's what the MIPS
assembler does.
<<mld-mips.spec>>=
constructors
  break7ifzero rt
    when { rt = 0 } is break(7)
    otherwise       is bne(rt, r0, L); nop(); break(7); L: epsilon
@
The other test, for overflow, is more complicated.
On a two's-complement machine,
integer division can overflow if it is signed, if the divisor is $-1$,
and if the dividend is the most negative integer---there aren't enough
bits to represent the result.
<<mld-mips.spec>>=
constructors
  break7ifoverflow rs, rt is
        addiu(r1, r0, -1);
        bne(rt, r1, L);   
        lui(r1, 0x8000);  
        bne(rs, r1, L);   
        nop();            
    	break(7);
    L : epsilon
@
Given these two tests, we can define safe versions of signed and
unsigned division and remainder.
<<mld-mips.spec>>=
constructors
  tested_divu rd, rs, rt
    is divu(rs, rt); nop(); break7ifzero(rt); mflo(rd); nop()
  tested_div rd, rs, rt
    is div(rs, rt);  nop(); break7ifzero(rt); break7ifoverflow(rs, rt); mflo(rd); nop()
  tested_remu rd, rs, rt
    is divu(rs, rt); nop(); break7ifzero(rt); mfhi(rd); nop()
  tested_rem rd, rs, rt
    is div(rs, rt);  nop(); break7ifzero(rt); break7ifoverflow(rs, rt); mfhi(rd); nop()
@

Applications can also play games to trade safety for efficiency.
Toolkit-generated encoding procedures
prevent the application from putting a value in a field that is too
small for it, so for example they won't let an application put 255 in
a 5-bit field. 
If an application writer wants to promise that the values of a field
will always fit, it can call the field ``unchecked,'' and the toolkit
will omit the check, but it will still mask out the high bits.
If the application can guarantee the high bits are zero, it can call a
field ``guaranteed,'' and the toolkit won't even do the masking.
Many applications, including {\tt mld}, include register allocators
that always assign proper values to the
fields denoting registers:
<<mips-regs.spec>>=
fieldinfo [ rs rt rd base fs ft fd ] is [ guaranteed ]
@ 

@
[[mld]] emits relocatable addresses into the data segments.
There is an argument for putting an address-emitting procedure into the
toolkit library, but then that procedure would have to deal with all
the details of generating a closure and so on when the value of the
address wasn't yet known.
Since the toolkit's generator already takes care of those details, it
is much easier simply to specify a constructor that emits an address.
We create a new token so we can distinguish a 32-bit data token
from a 32-bit instruction token, and
we choose a placeholder that is known to be an illegal address.
<<mld-mips.spec>>=
fields of addrtoken (32) addr32 0:31
placeholder for addrtoken is addr32 = 7
constructors
  emit_raddr reloc is addr32 = reloc
@
\subsection{Specifications to support validation}

We have validated this specification against the DEC-Ultrix MIPS
assembler.
To do that, we had to alter a little bit of syntax, and we had to
discard some constructors that this assembler didn't know about.
<<mips-names.spec>>=
assembly opcode
  jr                is j
  jalr              is jal
  mov               is move
  {sll,srl,sra}v    is $1
  tested_{*}        is $1
assembly operand
  [ rs rt rd base ] is "$%d"
  [ fs ft fd ]      is "$%s"
@ 
The machine instructions not recognized by the MIPS assembler
are discarded.
<<mips-check.spec>>=
discard 
  break lwc0 swc0 arith3.^"w" arith2.^"w" c.cond^"."^"w" cvt.s.s
  cvt.w.w cvt.d.d swc2 lwc2 swc3 lwc3 
@ 
On the MIPS, the checker's assembly output must include the
[[noreorder]] directive. 
<<mips-checker.s>>=
.set noreorder
